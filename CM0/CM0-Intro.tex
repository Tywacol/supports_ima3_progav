%\documentclass{beamer}
%\documentclass[12pt,serif]{beamer}
%\documentclass[12pt]{beamer}
\documentclass[12pt,svgnames]{beamer}
%\documentclass[handout,tikz,12pt,svgnames]{beamer}

\usepackage{CM-preamble}

\subtitle{\Huge Introduction et Rappel}

%\date{\today}
%\date{29 janvier 2016}
\date{CM0}

\begin{document}

\begin{frame}
	\titlepage
\end{frame}

%\begin{frame}
%  \frametitle{Outline}
%  \tableofcontents
%\end{frame}

\begin{frame}
	\frametitle{Moi\ldots \textit{\small (et ma décharge de responsabilité)}}
	\begin{itemize}
		\item Je suis étranger (hors UE)… et \textbf{j'ai un accent}
		\item Je me {\color{blue} trompe beaucoup} en français
		\begin{itemize}
			\item et en info, et en math, et \ldots
			\item n'hésitez pas à me corriger ou à me demander de répéter
		\end{itemize}
		\item \textit{Work In Progress}
		\begin{itemize}
			\item J'accepte des critiques (constructives mais pas que)\\
			et surtout des recommandations
			\item N'hésitez pas à poser des questions
			\item Je ne suis pas un expert du domaine
		\end{itemize}
	\end{itemize}
\end{frame}


\begin{frame}
	\frametitle{Conseils et règles}
	\begin{itemize}
		\item Installez \textbf{Linux}
		\begin{itemize}
			\item Très important pour votre carrière
			\item Linux est le gagnant de la course des systèmes d'exploitation (serveurs, routeurs, Internet, super calculateurs, satellites, voitures, Cloud, Android, ChromeBook, \ldots)
		\end{itemize}
		\item Utilisez \textbf{la ligne de commandes} (bash, zsh)
		\begin{itemize}
			\item Automatisabilité
			\item Rapidité, auto-complétion ($\Rightarrow$ touche tab)
			\item Travaillez à distance
		\end{itemize}
%		\item Quelques règles
		\item \textbf{\textit{No electronics policy}}
		\begin{itemize}
		\item \url{http://cs.brown.edu/courses/cs019/2018/laptop-policy.html}
		\item Je confisque les appareils $\ddot\smile$
		\item Pas de Facebook, pas de jeux vidéos, \ldots [CM/TD/TP]
		\end{itemize}		
		\item Ponctualité imposé, assiduité négociable
		\item Gagnez des Carambars
	\end{itemize}
\end{frame}



\begin{frame}
	\frametitle{Remarque}
	\Large Ce cours est très très très largement inspiré (i.e., copié) de ceux de Nathalie Devesa (Maître de Conférences à Polytech Lille), qui à son tour s'est inspiré de Bernard Carré et de Laure Gonnord.
\end{frame}

\begin{frame}
	\frametitle{Volume horaire et évaluation}
	\begin{columns}
	\column{.4\textwidth}
	\begin{block}{Volume horaire}
		\begin{itemize}
			\item 22h CM
			\item 14h TD
			\item 22h TP
			\item 10h ET/Projet
		\end{itemize}
	\end{block}{}

	\column{.6\textwidth}
	\begin{block}{Evaluation}
		\begin{itemize}
			\item DS (2h) --- 2 ECTS
			\item TP noté (2h) --- 1.5 ECTS
			\item Projet --- 1.25 ECTS
			\item Total: 4.75 ECTS
		\end{itemize}
	\end{block}{}	
	\end{columns}
\end{frame}


\begin{frame}
	\frametitle{Cont. de Programmation Structurée }
	\begin{itemize}
		\item Pr. Laurent Grisoni au S5
		\item Bases de l'algorithmique
		\begin{itemize}
			\item Pseudo-code, décomposition de problèmes en sous-problèmes, complexité
		\end{itemize}

		\item Bases de la programmation en C
		\begin{itemize}
			\item Variables, types de données, boucles, fonctions, tableaux/matrices, tris, pointeurs, paramètres variables
		\end{itemize}
		\item Outillage
		\begin{itemize}
			\item Compilation, éditeur de texte, ligne de commande, Linux, redirections
		\end{itemize}
	\end{itemize}
\end{frame}


\begin{frame}
	\frametitle{Programmation Avancé}
	\begin{block}{Objectifs}
		\begin{itemize}
		\item Organiser les données pour pouvoir y accéder rapidement et efficacement
		\item Avoir une connaissance de l'utilisation et l'implémentation des structures de données
		\item Estimer les coûts (mémoire \& temps)
		\end{itemize}
	\end{block}{}
	\begin{block}{Exemples de structures}
		\begin{itemize}
		\item Listes contiguës, listes chaînées, piles, queues, queues de priorités, tas, arbres, arbres binaires, arbres bicolores, tables de hachage, graphes, filtres de bloom, ...
		\end{itemize}
	\end{block}
\end{frame}


\begin{frame}
	\frametitle{Rappel --- Types de données \\\footnotesize (Ces valeurs peuvent varier selon l'architecture et le compilateur)}
	\setlength\extrarowheight{7pt}
	\scriptsize
%	\footnotesize
%	\tiny
%	{\fontsize{1cm}{0cm}
%	{\fontsize{5}{12}
	{
	\makebox[\textwidth][c]{
		%\texttt{%
		%\begin{tabularx}{1.1\linewidth}{ |X|r|c|r|c| }
		\def\arraystretch{0.7}%  1 is the default, change whatever you need
		\begin{tabularx}{1.07\linewidth}{ l|rcrc }
			%\hline
			%Type & Borne inférieure & Borne inférieure (formule) & Borne supérieure & Borne supérieure (formule)\\
			\textbf{Type} & \textbf{Min} & \textbf{Min form.} & \textbf{Max} & \textbf{Max formule}\\    \hline
			\Xhline{2\arrayrulewidth}
			char 	& -128 				& $-2^{7}$ 	& +127 		& $2^{7}-1$\\	%\hline
			unsigned char 	& 0 				& $0$ 		& +255 		& $2^{8}-1$\\ 	\hline
			short 			& -32 768 			& $-2^{15}$ & +32 767 	& $2^{15}-1$\\ 	%\hline
			unsigned short 	& 0 				& $0$ 		& +65 535 	& $2^{16}-1$\\ 	\hline
			int (16 bit)	& -32 768 			& $-2^{15}$ & +32 767 	& $2^{15}-1$\\ 	%\hline
			unsigned int 	& 0 				& $0$		& +65 535 	& $2^{16}-1$\\ 	\hline
			int (32 bit) 	& -2 147 483 648	& $-2^{31}$	& +2 147 483 647 & $2^{31}-1$\\ %\hline
			unsigned int 	& 0 				& $0$	 	& +4 294 967 295 & $2^{32}-1$\\ \hline
			long (32 bit)	& -2 147 483 648	& $-2^{31}$	& +2 147 483 647 & $2^{31}-1$\\ %\hline
			unsigned long 	& 0 				& $0$	 	& +4 294 967 295 & $2^{32}-1$\\ \hline
%			long long		& -9 223 372 036 854 775 808 	& $-2^{63}$ & +9 223 372 036 854 775 807	& $2^{63}-1$\\
%			unsig. long long& 0 				& 0 		& +18 446 744 073 709 551 615 				& $2^{64}-1$\\

			long (64 bit) 	& $-9.22337x10^{18}$ & $-2^{63}$ 	& $+9.22337x10^{18}$ 	& $2^{63}-1$\\
			unsig. long long& 0 				& 0 			& $+1.844674x10^{19}$& $2^{64}-1$\\ \hline
			long long		& $-9.22337x10^{18}$ & $-2^{63}$ 	& $+9.22337x10^{18}$ 	& $2^{63}-1$\\
			unsig. long long& 0 				& 0 			& $+1.844674x10^{19}$& $2^{64}-1$\\

		\end{tabularx}
	}
	}
	%}223 372 036 854 775 808
\end{frame}

\begin{frame}
	\frametitle{Rappel --- Taille des données}
%	\begin{figure}[ht]
	\inputminted[
	%	frame=lines,
	%	framesep=1.5mm,
	baselinestretch=0.4,
	fontsize=\footnotesize,
	linenos
	]
	{c}{../codes/size_ofs.c}
%	\captionof{listing}{size\_ofs.c}
	\vspace{-0.8cm} \center \small  \texttt{size\_ofs.c}
%	\caption{\small{Code size_ofs.c}}
%	\label{lst:firstListing}
%	\end{figure}
	%	\vspace{-1.7cm} \begin{center} \line(1,0){500} \end{center}  \vspace{-0.5cm}
\end{frame}

\begin{frame}
	\frametitle{Rappel --- Taille des données}
	\inputminted[
	%	frame=lines,
	%	framesep=1.5mm,
	baselinestretch=0.5,
	fontsize=\footnotesize,
	linenos,
	bgcolor=light-gray
	]
	{c}{../codes/size_ofs_output.txt}
%	\captionof{listing}{Output of size\_ofs.c}
    \vspace{-0.5cm} \center \small Sortie de \texttt{size\_ofs.c} (exemple)
\end{frame}

\begin{frame}
	\frametitle{Rappel --- Pointeurs \small (source: TD Pr. Grisoni)}
	
	\inputminted[
%	frame=lines,
%	framesep=1.5mm,
	baselinestretch=0.4,
	fontsize=\footnotesize,
	linenos
	]
	{c}{../codes/td6.c}
	%\caption{\small{Classe Animal}}
	%\label{lst:firstListing}
	%\end{figure}
%	\vspace{-1.7cm} \begin{center} \line(1,0){500} \end{center}  \vspace{-0.5cm}

	\pause
	\vspace{-1em}
	\inputminted[
	%	frame=lines,
%	framesep=1.5mm,
	baselinestretch=0.5,
	fontsize=\footnotesize,
	linenos,
	bgcolor=light-gray
	]
	{c}{../codes/td6_output.txt}
\end{frame}


\begin{frame}[fragile=singleslide]
	\frametitle{Rappel --- Pointeurs 2}
	\begin{adjustwidth}{-0.5em}{0em}
		\begin{minted}[
			mathescape=true,
			escapeinside=||,			
			%	frame=lines,
			%	framesep=1.5mm,
			%		baselinestretch=0.4,
					fontsize=\footnotesize,
					linenos
			]{c}
void main() {   
  int*    x;  // Alloue les pointeurs en mémoire 
  int*    y;  // (mais pas les valeurs pointés)

  x = malloc(sizeof(int));
      // Alloue un entier (valeur pointé),
      // et fait pointer x sur cette espace
    
  *x = 42; // Donne la valeur de 42 à l'espace pointée par x
           // (déreferencer x)

  *y = 13; // ERREUR (SEGFAULT)
           // il n'y a pas d'espace pointé en mémoire

  y = x; // Fait pointer y sur le même espace mémoire que x
    
  *y = 13; // Déréférence y et assigne 13
           // (espace pointé par x et y)
  free(x); // Libère l'espace alloué
}
		\end{minted}
	\end{adjustwidth}
\end{frame}


%%%%%%%%%%%%%%%%%%%%%%%%%%%%%%%%%%%%%%%%%%%%%%%%%%%%%%%%%%%%%%%%%%%%%%%%%%%%%%%%%
%\begin{frame}
%	\frametitle{}
%	\begin{itemize}
%		\item
%		\item
%		\begin{itemize}
%			\item
%		\end{itemize}
%		\item 
%		\begin{itemize}
%			\item
%		\end{itemize}
%	\end{itemize}
%\end{frame}

%
%\begin{frame}
%	\frametitle{Possible future directions}
%	\begin{block}{Automated detection or static analysis}
%		\begin{itemize}
%			\item Component coupling, Transactions
%		\end{itemize}
%	\end{block}{}
%%	\pause
%	\begin{block}{Request tracking}
%		\begin{itemize}
%			\item High-level concept for programming applications
%		\end{itemize}
%	\end{block}{}
%%	\pause
%	\begin{block}{Exception framework}
%		\begin{itemize}
%			\item Component-level vs. System-level exceptions
%			\begin{itemize}
%				\item Architecture invariants
%			\end{itemize}
%		\end{itemize}
%	\end{block}{}
%	\begin{alertblock}{Manage change}
%		\begin{itemize}
%			%\item Manage change
%			%\begin{itemize}
%				\item From $\textbf{Dynamic Dependencies}$ to $\textbf{Dynamic Architecture}$
%			%\end{itemize}
%		\end{itemize}
%	\end{alertblock}{}	
%\end{frame}


%\begin{frame}
%\begin{center}
%\Huge The End\\
%\Huge Thank You!
%\end{center}
%\end{frame}


\end{document}

